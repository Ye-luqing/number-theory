\documentclass[twoside,11pt]{article} 
\usepackage{amsmath,amsfonts,bm}
\usepackage{hyperref}
\usepackage{amsthm} 
\usepackage{amssymb}
\usepackage{framed,mdframed}
\usepackage{graphicx,color} 
\usepackage{mathrsfs,xcolor} 
\usepackage[all]{xy}
\usepackage{fancybox} 
% \usepackage{CJKutf8}
\usepackage{xeCJK}
\newtheorem{theorem}{定理}
\newtheorem{lemma}{引理}
\newtheorem{corollary}{推论}
\newtheorem*{exercise}{习题}
\newtheorem*{example}{例}
\setCJKmainfont[BoldFont=Adobe Heiti Std R]{Adobe Song Std L}
% \usepackage{latexdef}
\def\ZZ{\mathbb{Z}} \topmargin -0.40in \oddsidemargin 0.08in
\evensidemargin 0.08in \marginparwidth 0.00in \marginparsep 0.00in
\textwidth 16cm \textheight 24cm \newcommand{\D}{\displaystyle}
\newcommand{\ds}{\displaystyle} \renewcommand{\ni}{\noindent}
\newcommand{\pa}{\partial} \newcommand{\Om}{\Omega}
\newcommand{\om}{\omega} \newcommand{\sik}{\sum_{i=1}^k}
\newcommand{\vov}{\Vert\omega\Vert} \newcommand{\Umy}{U_{\mu_i,y^i}}
\newcommand{\lamns}{\lambda_n^{^{\scriptstyle\sigma}}}
\newcommand{\chiomn}{\chi_{_{\Omega_n}}}
\newcommand{\ullim}{\underline{\lim}} \newcommand{\bsy}{\boldsymbol}
\newcommand{\mvb}{\mathversion{bold}} \newcommand{\la}{\lambda}
\newcommand{\La}{\Lambda} \newcommand{\va}{\varepsilon}
\newcommand{\be}{\beta} \newcommand{\al}{\alpha}
\newcommand{\dis}{\displaystyle} \newcommand{\R}{{\mathbb R}}
\newcommand{\N}{{\mathbb N}} \newcommand{\cF}{{\mathcal F}}
\newcommand{\gB}{{\mathfrak B}} \newcommand{\eps}{\epsilon}
\renewcommand\refname{参考文献} \def \qed {\hfill \vrule height6pt
  width 6pt depth 0pt} \topmargin -0.40in \oddsidemargin 0.08in
\evensidemargin 0.08in \marginparwidth0.00in \marginparsep 0.00in
\textwidth 15.5cm \textheight 24cm \pagestyle{myheadings}
\markboth{\rm \centerline{}} {\rm \centerline{}}
\begin{document}
\title{\huge{\bf{整线性映射 $T:\mathbf{Z}^2\to \mathbf{Z}^2$是双射的充
      要条件 }}} \author{\small{叶卢
    庆\footnote{叶卢庆(1992---),男,杭州师范大学理学院数学与应用数学专业
      本科在读,E-mail:h5411167@gmail.com}}\\{\small{杭州师范大学理学院,浙
      江~杭州~310036}}} \date{}
\maketitle

% ----------------------------------------------------------------------------------------
% ABSTRACT AND KEYWORDS
% ----------------------------------------------------------------------------------------


\textbf{\small{摘要}:}若线性映射 $T:\mathbf{Z}^2\to \mathbf{Z}^2$ 对应
的矩阵中各个项都是整数,则称 $T$ 为从 $\mathbf{Z}^2$ 到 $\mathbf{Z}^2$
的整线性映射.本文给出了整线性映射 $T:\mathbf{Z}^2\to \mathbf{Z}^2$ 为
双射的充要条件.  \smallskip

\textbf{\small{关键词}:}裴蜀定理,整线性映射\smallskip


\vspace{30pt} % Some vertical space between the abstract and first section

% ----------------------------------------------------------------------------------------
% ESSAY BODY
% ----------------------------------------------------------------------------------------

\section{概念,记号与引理}
\label{sec:1}
\begin{enumerate}
\item $\mathbf{Z}^2$ 是如下集合:
$$
\mathbf{Z}^2=\{(m,n)|m,n\in \mathbf{Z}\}.
$$
\item 设 $T:\mathbf{Z}^2\to \mathbf{Z}^2$ 是线性映射.且设 $T$ 对应的矩
  阵为
$$
\begin{pmatrix}
  a&b\\
c&d
\end{pmatrix},
$$
也即,$\forall (m,n)\in \mathbf{Z}^2$,
$$
T((m,n))=\begin{pmatrix}
  a&b\\
c&d
\end{pmatrix}\begin{pmatrix}
  m\\
n\\
\end{pmatrix}.
$$
若 $a,b,c,d\in \mathbf{Z}$,则称 $T$ 为从 $\mathbf{Z}^2$ 到
$\mathbf{Z}^2$ 的整线性映射.
\item 裴蜀定理:若 $a,b\in \mathbf{Z}$ 且 $a,b$ 互素,则存在 $p,q\in
  \mathbf{Z}$,使得
$$
pa+qb=1.
$$
\end{enumerate}

\section{主要结论}
\label{sec:2}
我们现在来证明,
\begin{theorem}
若 $T:\mathbf{Z}^2\to \mathbf{Z}^2$ 是整线性映射,且 $T$ 对应的矩阵为
$$
\begin{pmatrix}
  a&b\\
c&d
\end{pmatrix},
$$
则 $T$是双射的充要条件是
$$
\begin{vmatrix}
  a&b\\
c&d
\end{vmatrix}=\pm 1.
$$
\end{theorem}
\begin{proof}[\textbf{证明}]
$\Leftarrow:$首先我们证明 $T$ 是单射,也即证明,当 $(m,n)\neq (p,q)$ 时,
$$
\begin{pmatrix}
  a&b\\
c&d
\end{pmatrix}\begin{pmatrix}
  m\\
n
\end{pmatrix}\neq \begin{pmatrix}
  a&b\\
c&d
\end{pmatrix} \begin{pmatrix}
  p\\
q
\end{pmatrix}.
$$
这由矩阵
$$
\begin{pmatrix}
  a&b\\
c&d
\end{pmatrix}
$$
的可逆性可以很容易推出.\\

其次我们证明 $T$ 是满射,也即证明,对于任意的
$(p,q)\in \mathbf{Z}^2$,都存在相应的 $(m,n)\in \mathbf{Z}^2$,使得
$T((m,n))=(p,q)$.我们来看方程组
\begin{equation}\label{eq:1}
\begin{cases}
  am+bn=p,\\
cm+dn=q
\end{cases},
\end{equation}
解得
\begin{equation}\label{eq:2}
\begin{cases}
  m=\frac{\begin{vmatrix}
    p&b\\
q&d
  \end{vmatrix}}{\begin{vmatrix}
    a&b\\
c&d
  \end{vmatrix}},\\
n=\frac{\begin{vmatrix}
    a&p\\
c&q\\
  \end{vmatrix}}{\begin{vmatrix}
    a&b\\
c&d
  \end{vmatrix}}.
\end{cases}
\end{equation}
而由于 
$$
\begin{vmatrix}
  a&b\\
c&d
\end{vmatrix}=\pm 1,
$$
因此,$m,n$ 在 $\mathbf{Z}$ 中都有唯一解.因此证明了 $T$ 是满射.\\

综上所述,$T$ 是双射.\\

$\Rightarrow:$设 $(m,n)\in \mathbf{Z}^2,(p,q)\in \mathbf{Z}^2$,我们来
看方程组 \ref{eq:1} 以及其解 \ref{eq:2}.假如
$$
\begin{vmatrix}
  a&b\\
c&d
\end{vmatrix}\neq \pm 1,
$$
那么 $b,d$ 必定不互素,否则根据裴蜀定理,可以选取合适的 $p,q$,使得
$$
\begin{vmatrix}
  p&b\\
q&d
\end{vmatrix}=1,
$$
这样就导致 $m$ 的整数解不存在,这与 $T$ 是双射矛盾.同理,可得 $a,c$ 必定
不互素,否则将会导致 $n$ 的整数解不存在.不妨设 $b,d$ 的最大公约数为
$k_1>1$,$a,c$ 的最大公约数为 $k_2>1$.设
$b=b'k_1,d=d'k_1$,$a=a'k_2,c=c'k_2$.则 $b',d'$ 互素,$a',c'$ 互素.我们
来看
$$
m=\frac{\begin{vmatrix}
    p&b'\\
q&d'
  \end{vmatrix}}{\begin{vmatrix}
    a&b'\\
c&d'
  \end{vmatrix}}.
$$
易得此时必然有
$$
\begin{vmatrix}
  a&b'\\
c&d'
\end{vmatrix}=\pm 1,
$$
否则容易由裴蜀定理得到 $m$ 无整数解这个矛盾.然而我们知道
$$
\begin{vmatrix}
  a&b'\\
c&d'
\end{vmatrix}=k_2 \begin{vmatrix}
  a'&b'\\
c'&d'
\end{vmatrix}\neq 1,
$$
因此,我们的假设错误,可见,
$$
\begin{vmatrix}
  a&b\\
c&d\\
\end{vmatrix}=\pm 1.
$$
\end{proof}
% BIBLIOGRAPHY
% ----------------------------------------------------------------------------------------
%
% ----------------------------------------------------------------------------------------
\end{document}








